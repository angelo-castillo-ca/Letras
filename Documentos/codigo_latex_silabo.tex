\documentclass[a4paper]{article}
\usepackage{hyperref}
\usepackage[left=3cm,right=3cm,top=2.5cm,bottom=2.5cm]{geometry}
\usepackage{graphicx}
\begin{document}
\begin{titlepage}
\begin{center}
\includegraphics[width=0.2\textwidth]{logo UNMSM.png}
\end{center}
\begin{center}
{\huge \bf Universidad Nacional Mayor de San Marcos}\\[0.3cm]
{\large \bf Universidad del Perú. Decana de América}\\[0.5cm]
\end{center}
\vspace{0.5cm}
\begin{center}
\large{FACULTAD DE LETRAS Y CIENCIAS HUMANAS}
\large{UNIDAD DE POSGRADO}
\end{center}
\vspace{0.5cm}
\begin{center}
{\large Maestria en Literatura con mencion en Literatura Peruana y LatinoAmericana}\\[0.3cm]
\end{center}
\vspace{0.10cm}
\begin{center}
{\huge \bf SILABO}\\[0.3cm]
\end{center}
\vspace{2cm}
\begin{flushleft}
Nombre de la asignatura: Seminario de Tesis I \\ 
Profesor responsable: Elizabeth Huisa Veria \\ 
Correo Electronico: ehuisav@unmsm.edu.pe \\ 
\end{flushleft}
\vspace{2cm}
\begin{center}
{\huge \bf 2022-I}\\[0.3cm]
\end{center}
\end{titlepage}
\newpage
\section{INFORMACION GENERAL}
1.1. Nombre de la asignatura :Seminario de Tesis I \\ 
1.2. Tipo de asignatura : Profundizacion o Investigacion \\ 
1.3. Profesor(a) : Elizabeth  \\ 
1.4. Programa : Maestria en Literatura \\ 
1.5. Mención : Literatura Peruana y LatinoAmericana \\ 
1.6. Código de asignatura : L76115 \\ 
1.7. Créditos : 6  \\ 
1.8. N° de horas semanales : 3 horas \\ 
1.9. N° de horas por semestre : 48 horas \\ 
1.10. Semestre académico : 2022-I \\ 
1.11. Duración : 16 semanas \\ 
1.12. Fecha de inicio : 2023-04-11 \\ 
1.13. Fecha de finalización : 2023-07-25 \\ 
1.14. Local y aula : Virtual \\ 
1.15. Horario : Virtual \\ 
\section{FUNDAMENTOS DE LA ASIGNATURA}
\subsection{Sumilla}
Es un seminario de investigación interdisciplinar. Reelabora el anteproyecto de tesis de
postulante a la Maestría. Reflexiona y discute las epistemes de las humanidades para el
desarrollo disciplinar. Conoce la metodología de investigación científica y de las humanidades.
Identifica y profundiza el tema-problema central de la tesis en una perspectiva de proyecto de
investigación teórico metodológico y de sustento académico. Indaga y sustenta los principales
antecedentes, las teorías y categorías que implica la investigación de tesis con una visión crítica
y académica. Redefine y planifica el proyecto de tesis de Maestría. Redacta, discute, defiende y
aprueba la versión final del proyecto de tesis e inscribe en el Registro de Proyectos de Tesis de la
UPG. Entregable: (1) Proyecto de Tesis de Maestría; (2) Resumen de primera ponencia sobre su
tesis a presentar en un evento académico nacional o internacional (de presentar ponencia esta
se valida con el entregable (3) del Seminario de Tesis II).)
\subsection{Competencia General}
Investiga y aporta nuevos conocimientos a las ciencias humanas
\subsection{Competencias Especificas}
\begin{itemize}
\item Definir el objeto de estudio de la tesis y reconocer su estructura
\item Establecer las preguntas de investigación y las hipótesis
\item Organizar la metodología de trabajo para elaborar el Plan de Tesis
\end{itemize}
\newpage
\begin{flushleft}
{\huge \bf ASIGNATURA}\\[0.3cm]
\end{flushleft}
\section{CONTENIDO TEMÁTICO}
\subsection{Unidad de aprendizaje I:("La investigación y los métodos de la historia del arte")}
\begin{table}[ht]
\centering
\begin{tabular}{|c|c|c|}
\hline
\textbf{Semana} & \textbf{Temas} & \textbf{Fecha} \\ 
\hline
Primera 
& \begin{minipage}[t]{10cm}
\begin{itemize}
\item Presentación del curso.
\item Metodología. Conceptos generales.
\end{itemize}
\end{minipage} & 11-04\\ 
\hline 
Segunda 
& \begin{minipage}[t]{10cm}
\begin{itemize}
\item Presentación del anteproyecto
\item Avance 1: elección del tema
\end{itemize}
\end{minipage} & 18-04\\ 
\hline 
Tercera 
& \begin{minipage}[t]{10cm}
\begin{itemize}
\item Lecturas:
\item Stastny, Francisco (1965) “Estilo y motivos en el estudio iconográfico. Ensayo de la metodología de la historia del arte” (5-19)
\item Barriga Tello, Martha (1992). “De los métodos y áreas de investigación en la Historia del Arte Peruano. Una propuesta” (49-63).
\item Victorio, Patricia (2010), “Re
\end{itemize}
\end{minipage} & 25-04 \\ 
\hline 
Cuarta 
& \begin{minipage}[t]{10cm}
\begin{itemize}
\item Panofsky, E. “Iconografía e iconología: introducción al arte del Renacimiento” (45-75)
\item Gombrich, E.H. (2003) “Magia, mito, metáfora” (184-211)
\item Barthes, Roland (1986). “La imagen” (11-77)
\item Avance 2: Búsqueda bibliográfica
\end{itemize}
\end{minipage} & 02-05 \\ 
\hline 
\end{tabular}
\end{table} 
\newpage
\subsection{Unidad de aprendizaje II:("Arte y cultura visual")}
\begin{table}[ht]
\centering
\begin{tabular}{|c|c|c|}
\hline
\textbf{Semana} & \textbf{Temas} & \textbf{Fecha} \\ 
\hline
Primera 
& \begin{minipage}[t]{10cm}
\begin{itemize}
\item Bryson, Norman (1991). “La imagen desde dentro y desde fuera” (81-98)
\item Bryson, Norman (1991). “Imagen, discurso y poder” (142-167)
\item Baxandall, Michael (1978) “El ojo de la época” (45-137)
\end{itemize}
\end{minipage} & 09-05\\ 
\hline 
Segunda 
& \begin{minipage}[t]{10cm}
\begin{itemize}
\item Burke, Peter (2008). “Cómo interrogar a los testimonios visuales”. 29-40.
\item Burke, Peter (2005). Visto y no visto. El uso de la imagen como documento histórico. Barcelona: Crítica. Cap. 1, 25-41 y Cap. 2, 43-57
\item Didi-Huberman, Georges (2011). “La historia del arte como disciplina anacrónica”. 29-97
\end{itemize}
\end{minipage} & 16-05\\ 
\hline 
Tercera 
& \begin{minipage}[t]{10cm}
\begin{itemize}
\item Belting, Hans (2007) “Medio-Imagen-Cuerpo” (13-70)
\item Mitchell, W.J.T. (2019). “Cuatro conceptos fundamentales de la ciencia de la imagen” (23-30) “La ciencia de la imagen” (31-44)
\item Wölfflin, E. (1952). “Forma cerrada y forma abierta” (177-221)
\end{itemize}
\end{minipage} & 23-05 \\ 
\hline 
Cuarta 
& \begin{minipage}[t]{10cm}
\begin{itemize}
\item White, Hayden (1992). El contenido de la forma. Narrativa, discurso y representación histórica. Buenos Aires: Paidós. Cap.2, 41-74, y Cap.3, 75-101
\item Crary, Jonathan (2008). Las técnicas del observador. Visión y modernidad en el siglo XIX. Murcia: Cendeac:
\item 15-46 Hang, Bárbara [y] Muñoz Agustina (com
\end{itemize}
\end{minipage} & 30-05 \\ 
\hline 
\end{tabular}
\end{table}
\newpage
\subsection{Unidad de aprendizaje III:("Asesorías de investigación")}
\begin{table}[ht]
\centering
\begin{tabular}{|c|c|c|}
\hline
\textbf{Semana} & \textbf{Temas} & \textbf{Fecha} \\ 
\hline
Primera 
& \begin{minipage}[t]{10cm}
\begin{itemize}
\item Presentación de avances y asesorías personalizada (Grupo 1)
\item Avance 4: Presentación del problema de investigación e
\item hipótesis y de la estructura de la tesis-Índice tentativo
\end{itemize}
\end{minipage} & 06-06\\ 
\hline 
Segunda 
& \begin{minipage}[t]{10cm}
\begin{itemize}
\item Presentación de avances y asesorías personalizada (Grupo 2)
\item Avance 4: Presentación del problema de investigación e hipótesis y
\item de la estructura de la tesis-Índice tentativo
\end{itemize}
\end{minipage} & 13-06\\ 
\hline 
Tercera 
& \begin{minipage}[t]{10cm}
\begin{itemize}
\item Presentación de avances y asesorías personalizada (Grupo 3)
\item Avance 4: Presentación del problema de investigación e hipótesis y
\item de la estructura de la tesis-Índice tentativo
\end{itemize}
\end{minipage} & 20-06 \\ 
\hline 
Cuarta 
& \begin{minipage}[t]{10cm}
\begin{itemize}
\item Presentación de avances y asesorías personalizada (Grupo 4)
\item Avance 4: Presentación del problema de investigación e hipótesis y
\item de la estructura de la tesis-Índice tentativo
\end{itemize}
\end{minipage} & 27-06 \\ 
\hline 
\end{tabular}
\end{table}
\newpage
\subsection{Unidad de aprendizaje IV:("Presentación final de proyectos")}
\begin{table}[ht]
\centering
\begin{tabular}{|c|c|c|}
\hline
\textbf{Semana} & \textbf{Temas} & \textbf{Fecha} \\ 
\hline
Primera 
& \begin{minipage}[t]{10cm}
\begin{itemize}
\item Exposición y discusión
\end{itemize}
\end{minipage} & 04-07\\ 
\hline 
Segunda 
& \begin{minipage}[t]{10cm}
\begin{itemize}
\item Exposición y discusión
\end{itemize}
\end{minipage} & 11-07\\ 
\hline 
Tercera 
& \begin{minipage}[t]{10cm}
\begin{itemize}
\item Exposición y discusión
\end{itemize}
\end{minipage} & 18-07 \\ 
\hline 
Cuarta 
& \begin{minipage}[t]{10cm}
\begin{itemize}
\item Entrega final de los proyectos.
\item Balance y reflexiones en torno a la elaboración del proyecto de
\item investigación
\end{itemize}
\end{minipage} & 25-07 \\ 
\hline 
\end{tabular}
\end{table}
{\huge \bf REFERENCIAS}\\[0.3cm]
\begin{itemize}
\item Barthes, Roland (1986). Lo obvio y lo obtuso. Imágenes, gestos, voces. Barcelona: Paidós.
\item Barriga Tello, Martha (1992). “De los métodos y áreas de investigación en la Historia del Arte Peruano. Una propuesta”. En: Letras, No 91, Lima. 49-63.
\item Baxandall, Michael (1978). Pintura y vida cotidiana en el Renacimiento. Barcelona: Gustavo Gilli.
\item Belting, Hans (2007). Antropología de la imagen. Buenos Aires: Katz
\item Bresciano, Juan Andrés (2004). Investigar en Humanidades. Montevideo: Psicolibros-Waslala.
\item Bryson, Norman (1991). Visión y pintura. La lógica de la mirada. Madrid: Alianza.
\item Burke, Peter (2008). “Cómo interrogar a los testimonios visuales”.  (\url{https://www.academia.edu/15204237/COMO\_INTERROGAR\_A\_LOS\_TESTIMONIOS\_VISUALES\_PETER\_BURKE})
\item Burke, Peter (2005). Visto y no visto. El uso de la imagen como documento histórico. Barcelona: Crítica.
\item Crary, Jonathan (2008). Las técnicas del observador. Visión y modernidad en el siglo XIX. Murcia.
\item Cruz Pedroni, Juan (2019). “La institucio
\end{itemize}
{\huge \bf Recursos electrónicos:}\\[0.3cm]
\begin{itemize}
\item Catálogo en línea de la Biblioteca Central de la UNMSM “Pedro Zulen”  (\url{https://unmsm.ent.sirsi.net/cl})
\end{itemize}
\section {ESTRATEGIAS METODOLÓGICAS}
\begin{itemize}
\item Desarrollo teórico del curso según el programa.
\item Clases magistrales.
\item Debates en torno a la estructu
\end{itemize}
\begin{itemize}
\item La evaluación es integral, continua y acumulativa. La nota promocional se obtendrá como resultado de:
\end{itemize}
\section {ESTRATEGIAS DE EVALUACIÓN}
\subsection {Modalidades de evaluación:}
\begin{itemize}
\item Asistencia y participación en el debate de las clases virtuales. La participación en cada clase es evaluada.
\item Exposición de las lecturas asignadas.
\item Cumplimiento de la tarea académica: presentación de la reseña comentada de cada lectura.
\end{itemize}
\subsection {Criterios de evaluación:}
\begin{table}[ht]
\centering
\begin{tabular}{|c|c|}
\hline
\textbf{Modalidades} & \textbf{Porcentaje} \\ 
\hline
1. Asistencia a clases virtuales y participación en el debate semanal & 25\% \\ 
\hline
2. Exposición de la lectura asignada & 50\% \\ 
\hline
3. Reseñas & 25\% \\ 
\hline
Total & 100\% \\ 
\hline 
\end{tabular}
\end{table}
\subsection {Obtención del promedio final:}
\begin{itemize}
\item Promedio de 1(25\%) +2 (50\%) +3 (25\%)
\item Nota aprobatoria mínima: 13
\end{itemize}
\subsection {Requisitos para aprobar la asignatura:}
\begin{itemize}
\item El estudiante con 30\% de inasistencias no será evaluado.
\end{itemize}
\begin{flushright}
Ciudad Universitaria, 06 de abril de 2022
\end{flushright}
\end{document}
